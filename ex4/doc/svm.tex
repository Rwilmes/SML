\newif\ifvimbug
\vimbugfalse

\ifvimbug
\begin{document}
\fi

\exercise{Support Vector Machines}
In this exercise, you will use the dataset \texttt{iris-pca.txt}. It is the same dataset used for Homework 3, but the data has been pre-processed with PCA and only two kind of flowers (`Setosa' and `Virginica') have been kept, along with their two principal components. Each row contains a sample while the last attribute is the label ($0$ means that the sample comes from a `Setosa' plant, $2$ from `Virginica').
(You are allowed to use built-in functions for computing the mean, the covariance, eigenvalues, eigenvectors and for quadratic programming.)
\begin{questions}

%----------------------------------------------

\begin{question}{Definition}{3}
Briefly define SVMs. What is their advantage w.r.t. other linear approaches we discussed this semester? 


\begin{answer}
SVMs try to find a hyperplane that separates the datasets. This is done by maximizing the distance from the plane to the closest data points of each class. A very important property of SVMs is that they are sparse, i.e. the classifier only depends on a very small number of support vectors. This makes classification efficient.	
\end{answer}
\end{question}

%----------------------------------------------

\begin{question}{Quadratic Programming}{5}
Formalize SVMs as a constrained optimization problem.

\begin{answer}\end{answer}
\end{question}

%----------------------------------------------

\begin{question}{Slack Variables}{5}
Explain the concept behind slack variables and reformulate the optimization problem accordingly. Without showing all the intermediate steps, write down the final solution of the problem.

\begin{answer}\end{answer}
\end{question}

%----------------------------------------------

\begin{question}{Kernel Trick}{4}
Explain the kernel trick and why it is particularly convenient in SVMs.

\begin{answer}\end{answer}
\end{question}

%----------------------------------------------

\begin{question}{Decision Boundary}{8}
Learn a SVM to classify the data in \texttt{iris-pca.txt}. Choose your kernel. Create a plot showing the data and the decision boundary. Attach a snippet of your code.

\begin{answer}\end{answer}

\end{question}

\end{questions}
