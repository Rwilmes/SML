\newif\ifvimbug
\vimbugfalse

\ifvimbug
\begin{document}
\fi

\exercise{Density Estimation}
In this exercise, you will use the datasets \texttt{densEstClass1.txt} 
and \texttt{densEstClass2.txt}. The datasets contain 2D data belonging
to two classes, $C_1$ and $C_2$.

\begin{questions}

%----------------------------------------------

\begin{question}{Gaussian Maximization Likelihood Estimate}{10}
Derive the ML estimate for the mean and covariance of the \textbf{multivariate} Gaussian distribution. Start your derivations with the function you optimize. Assume that you can collect i.i.d data. (Hint: you can find many matrix identities on the Matrix Cookbook and at \url{http://en.wikipedia.org/wiki/Matrix_calculus}.)

\begin{answer}

\end{answer}
{\LARGE TODO}
\end{question}


%----------------------------------------------

\begin{question}{Prior Probabilities}{2}
Compute the prior probability of each class from the dataset. 

\begin{answer}
Dataset 1 has 239 elements, dataset 2 has 761 elements, resulting in 1000 elements total. Therefore:\\
$p(C_1) = \frac{239}{1000}=0.239$, $p(C_2) = \frac{761}{1000}=0.761$
\end{answer}

\end{question}


%----------------------------------------------

\begin{question}{Biased ML Estimate}{5}
Define the bias of an estimator and write how we can compute it.
Then calculate the biased and unbiased estimates of the conditional distribution $p(x|C_i)$, assuming that each class can be modeled with a Gaussian distribution. Which parameters have to be calculated?
Show the final result and attach a snippet of your code.
Do not use existing functions, but rather implement the computations by yourself!

\begin{answer}

\end{answer}

\end{question}


%----------------------------------------------

\begin{question}{Class Density}{5}
Using the unbiased estimates from the previous question, fit a Gaussian distribution to the data of each class. Generate a single plot showing the data points and the probability densities of each class.
(Hint: use the contour function for plotting the Gaussians.) 

\begin{answer}

\end{answer}

\end{question}

%----------------------------------------------

\begin{question}{Posterior}{8}
In a single graph, plot the posterior distribution of each class $p(C_i|x)$ 
and show the decision boundary. For convenience, show the probabilities
in the log domain.

\begin{answer}

\end{answer}

\end{question}

%----------------------------------------------

\begin{question}{Bayesian Estimation}{15}{bonus}
State the generic case of a Bayesian treatment of the parameters $\theta$ used to model the data, .i.e., $p(\vec X \middle | \theta )$. 
Which are the advantages of being Bayesian? 
\\ Derive the estimate of the mean, assuming that the model is a Gaussian distribution with a fixed variance. (Hint: work approximately, complete the square.)


\begin{answer}
\end{answer}

\end{question}

\end{questions}
