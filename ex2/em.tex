\newif\ifvimbug
\vimbugfalse

\ifvimbug
\begin{document}
\fi

\exercise{Expectation Maximization}
 
In this exercise, you will use the datasets \texttt{gmm.txt}. It contains data from a Gaussian Mixture Model with four 2-dimensional Gaussian distributions.

\begin{questions}

%----------------------------------------------

\begin{question}{Gaussian Mixture Update Rules}{2}
Define the model parameters and the update rules for your model. 
Specify the E- and M-steps of the algorithm.

\begin{answer}
	The model parameters are $\mu_1$ ... $\mu_4$, $\sigma_1$...$\sigma_4$ and $\pi_1$ ... $\pi_4$ where $\mu_n$ is the mean, $\sigma_n$ the standard deviation and $\pi_n$ the prior of distribution n.
	
	In the E-Step, the posterior distribution of each component is calculated.\\
	$\alpha_{nj} = p(j|x_n) = \frac{\pi_j \mathcal{N}(x_n|\mu_j,\sigma_j)}{\sum_{i=1}^{4}\pi_i \mathcal{N}(x_n | \mu_i, \sigma_i)}$
	
	In the M-Step, the parameters are updated:\\
	$\mu_{n,new} = \frac{1}{N_j} \sum_{n=1}^{N} \alpha_{nj}x_n$, with $N_j = sum_{n=1}^{N} \alpha_{nj}$\\
	$\sigma_{j,new}^2 = \frac{1}{N_j} \sum_{n=1}^{N} \alpha_{nj}(x_n-\mu_{j,new})^2$
	
	$\pi_{j,new} = \frac{N_j}{N}$
	
\end{answer}

\end{question}

%----------------------------------------------

\begin{question}{EM}{18}
Implement the Expectation Maximization algorithm for Gaussian Mixture Models. Initialize your model uniformly. Generate plots at different iterations $t_i \in [1,3,5,10,30]$, showing the data and the mixture components, and plot the log-likelihood for every iteration $t_i=1:30$.

\begin{answer}
\end{answer}

\end{question}

\end{questions}
